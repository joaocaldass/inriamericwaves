\section{Introduction}

\indent Ce rapport de stage est divisé en deux parties, correspondant aux deux stages qui ont composé mon année de césure. Malgré les différentes thématiques abordées dans chacun d'eux, et le fait d'avoir être réalisés dans des différentes pays, ils ont plusieurs points en commun.

\indent La caractéristique commune la plus remarquable est que les deux stages se sont déroulés dans des domaines de la recherche en mathématiques appliquées. Plusieurs motivations m'ont guidé vers cette direction : des expériences précédentes (ayant déjà travaillé dans des projets d'initiation à la recherche scientifique au Brésil), l'étude poursuit à l'ENPC (les cours du département d’Ingénierie Mathématique et Informatique, le contact avec des professeurs chercheurs, des visites à des centres de recherche) et la carrière en recherche que j'envisage dans mon futur professionnel.

\indent Ces raisons m'ont conduit naturellement à un stage à Inria, dans son centre de recherche à Bordeaux. Le bon déroulement de ce premier stage m'ont motivé à continuer à travailler dans le contexte de l'Inria, et je suis allé à Santiago, au Chili, pour travailler à MERIC, un centre de recherche en énergie marine qui travaille en partenariat avec la Fundación Inria Chile.

\indent Aussi en conséquence des mes expériences au Brésil et à l'ENPC, et des mes motivations futures, les sujets des deux stages sont liés à la résolution numérique de problèmes de la mécanique des fluides. Par ailleurs, dans les deux stages j'ai travaillé sur des aspects mathématiques et numériques, même que dans des différentes proportions. 

\indent Néanmoins, malgré ces points communs, les sujets des deux stages ne sont pas directement liés. Ainsi, afin de rendre plus claire et organisée la contenus des deux travaux, ils sont présentés dans ce rapport dans des sections séparées.