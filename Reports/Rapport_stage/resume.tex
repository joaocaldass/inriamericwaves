\section*{Résumé}

\indent Ce rapport est divisé en deux parties, correspondant aux deux stages réalisés en année de césure.

\indent Le sujet du premier stage, réalisé à Inria Bordeaux, consiste dans l'étude et l'implémentation de méthodes d'adaptation de maillage, envisageant leur application à la résolution de problèmes de la mécanique des fluides.

\indent Le modèle d'adaptation considéré est basé exclusivement sur le mouvement des points du maillage, sans l'ajout ou suppression de noeuds ni la modification de sa connectivité (les relations de voisinage). L'objectif de cette adaptation est de concentrer les noeuds sur des régions du domaine où un raffinement plus grand est nécessaire (par exemple, dans les voisinages d'un objet ou dans les régions où la vitesse du fluide présente des forts gradients), en permettant des calculs plus précis avec le même coût computationnel.

\indent On commence par la présentation des aspects théoriques du modèle et ensuite on l'implémente, en utilisant une méthode d'éléments finis. Une bibliothèque en C a été développé au long du stage afin de permettre l'incorporation du modèle à des codes pour la mécanique des fluides, en deux et trois dimensions. On présente plusieurs tests réalisés pour la validation du modèle et de la bibliothèque, bien comme pour valider le couplage entre l'adaptation à des surfaces et l'adaptation à des variables physiques.

\indent Le deuxième stage, réalise au Chili (MERIC/Inria Chile), a eu comme principal objectif l'étude et l'implémentation de méthodes de décomposition de domaine (DDMs), appliquées à des modèles de propagation des ondes.

\indent On présente d'abord l'étude de deux de ces modèles, l'équation de KdV et les équations de Serre, qui prennent en compte des phénomènes non linéaires et dispersives. Pour les deux modèles, on propose et valide une résolution numérique avec une méthode de \emph{splitting}, en séparant les termes d'advection et les termes de dispersion.

\indent Ensuite, on présente le contenu concernant les méthodes de décomposition de domaine, appliquées à une équation dispersive (l'équation de KdV linéarisée sans le terme d'advection). On fait initialement une étude des conditions aux limites transparentes pour cette équation, et on propose et valide des approximations simples pour ces conditions, qui sont ensuite utilisées comme conditions à l'interface entre les subdomaines lors de l'implémentation d'une DDM. Un procès d'optimisation est réalisé permettant l'obtention de la méthode avec la convergence la plus rapide vers la solution du problème calculée dans le monodomaine.

\indent \textbf{Mots-clé:} adaptation de maillage, méthode d'éléments finis, méthode de décomposition de domaine, conditions aux limites transparentes, équation de KdV, équations de Serre
