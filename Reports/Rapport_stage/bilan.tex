\part*{Bilan personnel}
\addcontentsline{toc}{part}{Bilan personnel}

\indent J'avais choisi ces deux stages en espérant avoir des missions et expériences correspondantes à mes perspectives professionnelles : en envisageant une carrière dans la recherche en mathématiques appliquées, plus spécifiquement sur des méthodes numériques pour des équations différentielles, avec des applications en des problèmes de la mécanique des fluides (des thématiques que m'intéressent beaucoup et avec lesquelles j'avais déjà une expérience précédente). Maintenant, après la conclusion de l'année de césure, je peux affirmer que ces deux stages ont confirmé et renforcé ces perspectives et ma passion pour la recherche scientifique, en me donnant encore plus envie de poursuivre cette carrière.

\indent J'ai eu l'occasion de connaître plus profondément les plus variés aspects du monde de la recherche. D'abord, le fait de qu'il y a toujours quelque chose de nouveau à faire, à corriger ou à découvrir. Quand on se pose un objectif, pour y arriver il faut beaucoup étudier, écrire, programmer, tester et déboguer. Et quand on y arrive, on s'aperçoit qu'on peut toujours passer à des nouveaux cas, à des modèles plus complexes et comparer avec ce qui a été déjà fait par des autres scientifiques. Par ces mêmes raisons, j'ai pu vérifier que le travail dans la recherche se caractérise pour avoir, en citant Antoine Rousseau, un rythme de production complètement inconstant : parfois j'ai avancé très rapidement, mais j'ai aussi passé des semaines pour résoudre des petits problèmes. Dans tous les cas, c'était toujours des motivations pour poursuivre le travail.

\indent Un autre très important aspect que j'ai connu dans les deux stages est la multidisciplinarité qui peut se développer dans la recherche scientifique. À Bordeaux, des chercheurs et des étudiants de plusieurs domaines composaient l'équipe CARDAMOM, et, à Santiago, j'étais fortement impressionné quand j'ai connu toutes les lignes de recherche dans MERIC pour l'étude de la production d'énergie marine, dont la plupart je ne connaissais pas auparavant. Ainsi, je me suis rendu compte que mon travail, dans les deux stages et aussi dans le futur, n'est qu'une petite partie de ce qui peut être étudié dans la recherche.

\indent Un dernier point à remarquer sur ce contact avec la recherche scientifique, et qui constitue une des principales raisons qui ont confirmé mon intéresse dans ce domaine, est la très bonne ambiance de travail. Mes orientateurs et mes collègues d'équipe étaient toujours prêts à m'écouter et à m'aider et enseigner quand j'avais besoins. J'ai eu toujours l'occasion de proposer mes idées et de guider mon travail selon mes préférences.

\indent Dans ce qui concerne les expériences et connaissances techniques acquises dans les stages, j'ai beaucoup appris et j'ai renforcé des compétences et connaissances, sur des aspects mathématiques et numériques, en complémentant ce que j'avais appris à l'École des Ponts.

\indent À Bordeaux, j'ai travaillé à fond avec des modèles d'adaptation de maillage, qui étaient complètement nouveaux pour moi. En implémentant ces modèles avec des méthodes d'éléments finis, j'ai pu me familiariser à ce type de méthode, y compris ses aspects théoriques (la dérivation de la méthode à partir de la formulation variationnelle du problème) et pratiques (comme le calcul des éléments de la matrice du système linéaire, le stockage de la matrice creuse, le traitement des entités géométriques en deux et trois dimensions, etc.). Finalement, la création d'une bibliothèque m'a donné une forte expérience concernant le développement logiciel, la structure et spécificités de la langage C, les aspects liés à la compilation des programmes et le travail en utilisant Git pour gérer les versions et travailler en équipe.

\indent À Santiago, même en travaillant avec des sujets avec lesquels j'étais déjà plus familiarisé, j'ai eu un premier contact avec des nouveaux concepts, comme les modèles de propagation d'ondes qu'on a considéré et les études des conditions aux bords transparentes et les méthodes de décomposition de domaine. Par rapport au stage à Bordeaux, où mes tâches étaient plutôt numériques, je considère qu'à Santiago j'ai travaillé de façon plus équilibré entre les côtés mathématique et numérique, ce qui aura certainement une grande importance dans la suite de ma formation académique.

\indent Dans tous les cas, le contact et les discussions avec les orientateurs, les étudiants qui intégraient mes équipes de travail et d'autres chercheurs ont été essentielles pour consolider les connaissances acquises dans les stages.

\indent Toujours concernant les compétences acquises, je fais une remarque spéciale sur la production de textes scientifiques. Au long des deux stages, j'ai constamment rédigé des rapports pour enregistrer et organiser les achèvements et les tests et conclusions réalisés. Notamment, dans le deuxième stage j'ai écrit mon premier papier scientifique, qu'on a envoyé pour sa publication, et j'ai connu les plusieurs difficultés impliquées dans cette tâche : l'organisation, sélection et présentation de l'information, en pensant que texte sera lu par d'autres personnes, possiblement pas familiarisés avec le sujet; les constantes révisions; et le propre fait, aussi en citant Antoine Rousseau, ``qu'un papier n'est jamais fini".

\indent D'un point de vue moins technique, mais également important, le déroulement des stages en différentes villes et pays a complété l'expérience de l'année de césure. Je remarque notamment l'exercice au niveau linguistique : dans les deux stages, j'ai eu l'occasion de développer mon français et mon anglais, et aussi l'espagnol au Chili. Par ailleurs, dans le deuxième stage, j'ai pu aussi connaître un autre mode de vie et une autre culture.

\indent Finalement, en tenant compte des discussions et \emph{feedbacks} faites au long et à la fin des stages par mes orientateurs et les autres membres des équipes, je crois que j'ai apporté des importantes contributions à ses activités de recherche, ce qui est très motivateur et certainement me stimulera à continuer à travailler dans ce domaine. À Bordeaux, on a beaucoup avancé sur la bibliothèque d'adaptation de maillage, qui peut être déjà être utilisé dans des codes de mécaniques de fluides, étant ainsi utiles aux chercheurs et thésards de l'équipe; et au Chili, on est arrivé à des très bons résultats pour la décomposition de domaine appliquée à un des modèles de propagation  d'onde, ce qui peut guider les prochaines pas de MERIC dans la ligne de la modélisation mathématique pour l'énergie marine.