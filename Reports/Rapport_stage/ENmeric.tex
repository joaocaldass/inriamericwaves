\section*{MERIC (Marine Energy Research \& Innovation Center)}

\indent Inria Chile shares its offices at Santiago with MERIC, an excellency center whose objective is the research, technological development and innovation concerning the marine energy, in order to create a Chilean and international reference in this subject. Created by the french enterprise of naval defence DCNS , MERIC is funded by CORFO and the Chilean Department of Energy and runs its activities in partnership with Inria Chile, Chilean universities (Pontificia Universidade de Chile and Universidade Austral de Chile) , energy companies (Enel Green Power Chile and Chilectra) and institutes for promoting the innovation (Fundación Chile).


\indent

\indent

\begingroup
\centering
\includegraphics[scale=.3]{figures/logos/meric.png}
\captionof{figure}{MERIC's logo}
\endgroup


\indent The marine energy has a great importance in the Chilean and the global contexts. For Chile, with its coastline of over 6000 km in length, the tidal and wave movements create a huge energetic potential, being a strategic issue for the country. Moreover, Chile has special characteristics concerning the extremal natural phenomena : the intense seismic activity in the region make the earthquakes and tsunamis to be a reality in the Chilean history and quotidian. This fact requires specific studies in order to allow a riskless and sustainable production of marine energy.

\indent In a global point of view, the marine energy, for being a renewable energy source, is one of the main characters in the construction of the sustainable development. Nevertheless, it is yet a very new theme of study, requiring many efforts and researches for advancing and reducing the costs of the involved technologies, in order to allow its effective implementation and the replacement of traditional and limited energy sources. Additionally, it is essential to take into account that the promotion of the alternative energy sources must always consider all the environmental and social consequences of its production, looking for minimising the impacts on the habitats, its lifeforms and the human populations.

\indent It is in this context that MERIC develops its work, which is strongly multidisciplinary, encompassing the most varied subjects related to the production of marine energy, grouped in two main research lines : ``Sites development and knowledge of the Chilean environment" and ``Development of technologies in Chile".

\indent The first research line focuses in the study of potential sites for producing the marine energy in Chile. This study includes physical, chemical, biological and social aspects, under projects concerning, for example, the characterisation of these potential sites, the marine corrosion, the deposition of biological material (biofouling), the marine mammals, the assessment of social and environmental impacts and the mathematical modelization for the production of marine energy.  

\indent The second research line studies the installation, handling and use of technologies for producing marine energy, with projects concerning the technology development, its adaptation to the Chilean environment (regarding especially the natural risks), the development of a tests center for validation the research on the marine energy, the study of the state-of-art of the Chilean legislation concerning this subject, and the design of desalination plants.


\indent My work in this internship composed the first research line, in the project ``Advanced modeling for marine energy". Therefore, my mission was only a small piece among all the aspects linked to the production of marine energy and studied by MERIC.

