\section{Introduction}

\indent La ligne d'investigation de MERIC dans laquelle j'ai travaillé, ``Modélisation avancée pour l'énergie marine'', a comme principaux objectifs le ``couplage de modèles de petites résolutions spectrales a des grandes résolutions dans le domaine du espace-temps'' et ``la modélisation de grande résolution spatiale et temporale proche de la côte''\footnote{http://www.meric.cl/que-hacemos/}.

\indent Les sujets étudiés dans ce stage constituent une introduction à la recherche envisageant ces objectifs et peuvent être divisés en deux branches principales: la première consiste dans l'étude, implémentation numérique et simulation de modèles de propagation des ondes d'eau; la deuxième, plus directement liée à la ligne de recherche de MERIC, envisage l'étude, développement et application numérique de méthodes de décomposition de domaine (DDMs) pour la résolution de tels modèles.

\indent Concernant le premier point, on a étudié plusieurs modèles qui prennent en compte les effets non linéaires et dispersives dans la propagation des ondes. Cet étude s'est focalisé notamment sur l'équation de KdV et l'équation de Serre. En considérant les objectifs de la recherche développé par l'équipe, l'étude de différentes modèles mathématiques a une grande importance : premièrement, chacun d'eux est dérivée à partir d'hypothèses différents, et, ainsi, ses domaines de validité physique sont différents. Des critères comme les caractéristiques de l'onde (amplitude et longueur), la profondeur de l'eau et le fond océanique guident le choix d'un ou autre modèle. Dans le contexte de la production de l'énergie marine, par exemple, la distance à la côte est un facteur très important, et le choix du bon modèle est essentiel pour que les variables d'intérêt, comme la vitesse et l'amplitude des ondes, peuvent être calculées de manière fiable.

\indent Par ailleurs, en raison des différentes complexités de chaque modèle, les approches nécessaires pour son étude dans le contexte des méthodes de décomposition de domaine sont aussi différents. Dans ce stage, cet étude concernant l'équation de KdV a été bien avancé (entraînent même l'élaboration d'un papier scientifique qu'on a envoyé pour son publication; ce papier est aussi disponible sur la plateforme HAL \cite{joao2016}). Dans les dernières semaines du stage on a débuté l'étude de la résolution des équations de Serre en utilisant des DDMs, mais encore sans des résultats pour présenter dans ce rapport.

\indent Les DDMs consistent en diviser un domaine computationnel en plusieurs subdomaines et résoudre le problème en chacun d'eux. Pour que la solution du problème décomposé soit la même que la solution du problème calculé dans le monodomaine, il est essentiel qu'on définisse des conditions appropriées sur l'interface entre les subdomaines.

\indent Ainsi, dans la deuxième ligne de notre travail, on a d'abord fait un étude des conditions aux limites transparentes (TBCs) pour l'équation de KdV, en tenant compte que ces types de conditions aux bords jouent un rôle essentiel dans les méthodes de décomposition de domaine. Afin d'introduire ce sujet, on a suivi au début une approche plus simple, en optimisant des conditions aux bords classiques que fournissaient les conditions les plus proches des TBCs pour l'équation de KdV.

\indent Après, on s'est concentré dans l'étude des TBCs exactes pour l'équation de KdV linéarisée, en se basant sur les études développées par \cite{zheng2008} et \cite{besse2015}. Étant non locaux en temps, ces TBCs exactes ne peuvent pas être implémentées en pratique; alors notre but était de les proposer des approximation simples, qu'on a ensuite appliqué à une méthode de décomposition de domaine.

\indent À ce moment, il est nécessaire de clarifier et délimiter nos objectifs et les différences par rapport aux objectifs de \cite{zheng2008} et \cite{besse2015}. Pour cela, on fait une brève description des sources d'erreur et incertitudes liées aux simulations numériques de modèles physiques.

\indent De façon générale, ces erreurs peuvent être classifiées en des erreurs de modélisation conceptuelle et des erreurs numériques \cite{roache1997}. Dans le premier groupe, il y a les assomptions de modélisation conceptuelle (pour le phénomène physique et les conditions aux bords) et les incertitudes dans la géométrie, les données initiales, les données aux bords et les paramètres qui jouent un rôle dans le modèle \cite{roache1997,balagurusamy2008}. Dans ce qui concerne les erreurs numériques, on peut citer celles liées à la finitude do domaine computationnel, les erreurs temporales et les erreurs spatiales dues à la discrétisation des équations \cite{karniadakis1995,roache1997} et d'autres possibles sources d'erreur liées à la méthode numérique, comme dans des processus itératifs (comme est le cas de la DDM implémentée ici).

\indent L'erreur totale de simulation numérique est une somme des contributions de chacune de ces sources. Ainsi, la connaissance et quantification d'eux sont essentielles pour améliorer la description numérique du procès physique et, dans ce contexte, l'étude séparée de chaque contribution a une grande importance.

\indent Parmi les erreurs mentionnées ci-dessus, \cite{zheng2008} et \cite{besse2015} cherchaient à minimiser celle liée à la finitude du domaine computationnel. En fait, en tronquant le domaine avec l'introduction de bords artificielles, il faut utiliser des conditions aux bords additionnelles, qui doivent être choisies proprement afin d'assurer la stabilité et la précision du problème aux valeurs initiales et aux bords tronqué \cite{zheng2008}. Même en utilisant des approches différentes, les deux auteurs envisageaient à construire des conditions aux limites absorbantes (ABCs), qui simulent l'absorption d'une onde sortant du domaine computationnel, ou des TBCs, qui assurent que la solution approximée dans le domaine computationnel coïncide avec la solution de tout le domaine.

\indent Ainsi, notre travail ne doit pas utiliser la même solution de référence que celle utilisée par \cite{zheng2008} et \cite{besse2015} : pour valider leurs approches, ils comparent les solutions approximées avec la solution exacte dans tout le domaine. En revanche, notre solution de référence sera la solution approximée calculée dans le monodomaine computationnel. En suivant le principe d'étudier séparément chaque type d'erreur numérique, on ne cherche pas à minimiser l'erreur due à l'introduction de bordes externes au domaine computationnel, mais seulement l'erreur due 'a l'introduction d'une interface entre les subdomaines.

\indent

\indent Le rapport concernant ce deuxième stage est organisé de la façon suivante : dans les deux premières sections, on présente les modèles de propagation des ondes étudiés, l'équation de KdV (section \ref{sec:KdV}) et les équations de Serre \ref{sec:Serre}, et les détails de son implémentation numérique et la validation de cette implémentation. Ensuite, on passe à l'étude des conditions aux limites transparentes :  Dans la section \ref{sec:TBC}, on présente les aspect théoriques concernant les TBCs et on fait une étude introductoire avec l'équation de KdV; et dans la section \ref{sec:TBCKdV}, on propose des approximations pour les TBCs exactes de l'équation de KdV linéarisée. Finalement, dans la section \ref{sec:DDM}, on présente la théorie sur les méthodes de décomposition de domaine et on propose et implémente une pour résoudre l'équation de KdV linéarisée.