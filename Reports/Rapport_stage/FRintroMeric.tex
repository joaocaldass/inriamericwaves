\section{Introduction - stage à Meric/Inria Chile}

\indent Le travail développé dans le stage à Meric/Inria Chile a été composée par deux lignes principaux : l'étude de modèles dispersives et nonlinéaires pour la propagation d'ondes; et l'étude de méthodes de décomposition de domaines (\emph{Domain Decomposition Methods - DDM}) pour la résolution de PDEs. Évidemment, la suite naturelle de ce travail serait la jonction de ces deux lignes.

\indent Dans la première ligne, plusieurs modèles ont été étudiés, chacun correspondant à des différentes hypothèses physiques et ranges de validité. D'abord, on a étudié les équations de KdV et de BBM, qui sont modèles plus simples et qui sont valides pour des ondes de petite amplitude et grande longueur. Ensuite et jusqu'à la fin du stage, on a concentré nos efforts sur les équations de Serre, un modèle plus complexe, constitué par un système de deux équations, étant valide pour des ondes dans des eaux peu profondes.

\indent Le travail sur la deuxième ligne a commencé par l'étude des conditions aux limites transparentes (\emph{Transparent Boundary Conditions - TBC}), qui sont étroitement liées aux méthodes de décomposition de domaines. 

 