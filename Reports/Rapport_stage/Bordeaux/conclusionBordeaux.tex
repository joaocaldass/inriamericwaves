\section{Conclusion}

\indent Le travail réalisation au cours de ce premier stage a consisté dans l'étude et implémentation d'un modèle d'adaptation de maillage, basée exclusivement sur le mouvement des noeuds (\emph{i.e.}, sans modification du nombre de points et de la connectivité du maillage). L'objectif "concret" de ce travail était le développement d'une bibliothèque en C, appelée FMG, envisageant l'incorporation de ces modèles à des codes de résolution de problèmes de la mécanique de fluides.

\indent Étant réalisé dans deux lignes concomitantes et complémentaires, correspondant à l'étude de la bibliographie sur le sujet et à l'implémentation numérique, à la fin du stage ce travail se trouvait dans un état très avancé. En partant des modèles les plus classiques pour la détermination du mouvement des points du maillage, plusieurs tests ont été réalisées et des modifications ont été proposées et validées, en résultant dans une bibliothèque qui réalise l'adaptation 2D de forme robuste, relativement rapide et plus accessible à l'utilisateur final, qui n'est pas forcement habitué aux détails des modèles d'adaptation. Notamment, on peut définir les zones de raffinement du maillage à partir des tailles désirées pour chaque élément du maillage, ce qui a ainsi un sens physique très intuitive.

\indent Un autre important résultat dans ce stage a été le couplage de l'adaptation à des surfaces dans le domaine (\emph{i.e.}, à des fonctions \emph{Level Set}) et de l'adaptation physique (par exemple, à la vitesse calculée lors de la résolution des équations de Navier-Stokes). Cet couplage a été validé avec un processus itérative, consistant des successives adaptations et résolutions du problème de la mécanique des fluides, qui a permis l'obtention de solutions à chaque fois plus précises.

\indent Finalement, on peut mentionner d'autres éléments de travail qui, même étant encore à finaliser, ont été démarrés et constituent la suite naturelle de ce qui a été déjà réalisé. Par exemple, l'adaptation 3D, qui a été testée et validée dans des cas simples, mais pour laquelle des questions comme le coût de calcul et le croisement des points de maillages doivent être étudiés avec plus de détails; et l'adaptation à des cas non-stationnaires.

  