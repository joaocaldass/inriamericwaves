\section{Le modèle}
\label{sec:modele}

\subsection{Description du problème}

\indent On fait la distinction entre deux domaines : 

\begin{itemize}
  \item \textbf{Domaine physique ou réel } (\(\vecx =(x,y)\)) : domaine déformable, noté \(\dom\);
  \item \textbf{Domaine computationnel ou de référence } (\(\vecxi =(\xi,\eta)\)) : domaine fixé, noté \(\domRef\)
\end{itemize}

\indent On cherche une fonction 

\begin{equation*}
  \vecx = \vecx(\vecxi)
\end{equation*}

\indent Dans le modèle utilisé, on considère que la position \(\vecx\) des noeuds du maillage est régie par l'équation

\begin{equation*}
  %\label{eq:laplacien}
  \nabref \cdot \left( \omega \nabref \vecx \right) = 0
\end{equation*}

\noindent où \(\nabref\) est le gradient par rapport aux coordonnées de référence et \(\omega\) est une fonction de \(\vecx\) qui contient l'information qui déterminera le mouvement des noeuds. Dans le travail développé au cours du stage, on a implémenté deux modèles différents pour le calcul de cette fonction : 

\begin{enumerate}
	\item Dans un premier moment, on a implémenté le modèle utilisé dans \cite{arpaia} : en supposant qu'on fait l'adaptation par rapport à une fonction \(u\), \(\omega\) est donné par l'expression

	\begin{equation*}
  		\omega(\vecx) = \sqrt{1 + \alpha ||\nabref u(\vecx)|| + \beta ||H(u)(\vecx)||}
	\end{equation*}

	\indent $\nabref u(\vecx)$ et \(H(u)\) sont respectivement le gradient et le hessien de \(u\) par rapport aux variables spatiales de référence, et \(\alpha\) et \(\beta\) sont des paramètres choisis par l'utilisateur. Pour que ce choix soit moins dépendant du problème, on considère les gradients et les hessiens normalisés.
	
	\item On a ensuite utilisé un modèle où on fournit directement à chaque noeud \(i\) la taille de maille \(\hdes\) qu'on désire, selon la formulation présentée par \cite{askes} : 
	
	\begin{equation}
		\label{eq:omega2}
		\omega(\vecx) = \frac{1}{\hdes(\vecx)}
	\end{equation}
	
	\indent La façon dont on calcule les tailles désirées dépende du type d'adaptation qu'on fait (adaptation à une fonction Level Set ou à une solution physique), comme on précisera dans les sections suivantes de ce rapport.
	
	
\end{enumerate} 

\indent Pour que le problème soit bien posé, il faut définir des conditions aux bords appropriées : 

\begin{equation}
	\label{eq:systeme}
	\begin{cases}
  		\nabref \cdot \left( \omega \nabref \vecx \right) = 0 \ \ dans \ \ \domRef \\
  		\vecx = \vecg \ \ sur \ \ \bordRef^D \\
  		\nabref \vecx \cdot \vecn = 0 \ \ sur \ \ \bordRef^N 
	\end{cases}
\end{equation}

\indent Ainsi, les conditions aux limites utilisées sont de deux types, de Dirichlet et de Neumann, définies sur des parties disjointes du bord, (\(\bordRef^D\) et \(\bordRef^N\), respectivement). Pour les conditions de Dirichlet, on impose \(\vecg = \vecxi\), indiquant que les points de \(\bordRef^D\) ne doivent pas bouger (ce qu'on impose, par exemple, dans les coins d'un domaine rectangulaire). En revanche, les conditions de Neumann (imposées par exemple dans les côtés du domaine rectangulaire), indiquent que les points de \(\bordRef^N\) doivent glisser sur le bord, i.e., bouger parallèlement à lui (de façon que, dans notre exemple, le domaine reste toujours rectangulaire).

\indent La formulation faible du problème, avec une fonction test  \(v \in H_0^1(\domRef)\), s'écrit comme

\begin{equation}
	\label{eq:faible}
	0 = \iDom{v\nabref \cdot \left( \omega \nabref \vecx \right)} = -\iDom{\omega \nabref v \cdot \nabref \vecx} + \iBord{v\omega\nabref \vecx \cdot \vecn} 
\end{equation}

\indent Les conditions aux bords définies en \eqref{eq:systeme} annulent le dernier terme en \eqref{eq:faible}, et on arrive ainsi à 

\begin{equation*}
	\iDom{\omega \nabref v \cdot \nabref \vecx} = 0
\end{equation*}




\subsection{Discrétisation en éléments finis}

\indent On utilise une discrétisation en élément finis P1, avec une base de fonctions linéaires par morceaux \(\{\phii\}\) telles que $\phii(\vecx_i) = 1$ et $\phii(\vecx_j) = 0,\ \ i \neq j$, pour tous les $N$ noeuds du maillage. Ainsi, \(\vecx\) et la fonction test \(v \in H_0^1(\dom)\) se discrétisent sous la forme

\begin{equation}
  \label{eq:u_disc}
  \begin{gathered}
  \vecx_h = \sum_{j=1}^N{\vecx_j\phij} = \sum_{j=1}^N{ \left( \begin{array}{c}  x_j \\ y_j \end{array} \right)    \phij} \\
  v_h = \sum_{i=1}^N{v_i\phii} 
  \end{gathered}
\end{equation}

\noindent où $x_i$, $y_i$ et $v_i$ sont les valeurs des coordonnées spatiales et de la fonction test sur le noeud $i$. 

\indent En utilisant \eqref{eq:u_disc} dans \eqref{eq:faible}, on arrive à

\begin{equation*}
	\sum_{j=1}^N{ \left[  \left( \iDomh{ \omega \nabref \phii \cdot \nabref \phij }  \right)  \vecx_j \right] } = 0 \ \ \forall i \in \{1,...,N\}
\end{equation*}

\indent On voit ainsi que la discrétisation en éléments finis se ramène à la résolution de deux systèmes linéaires indépendants et de la même forme, un pour les coefficients \(\{x_j\}\) et l'autre pour \(\{y_j\}\) : 

\begin{equation}
	\label{eq:syst_final}
	\begin{cases}
		Kx = 0 \\
		Ky = 0
	\end{cases}
\end{equation}

\noindent où les éléments de la matrice \(K\) ont la forme

\begin{equation}
  k_{ij} = \iDomh{ \omega \nabref \phii \cdot \nabref \phij }
\end{equation}

\indent On propose une méthode itérative pour la résolution de ces systèmes linéaires, comme décrit dans la sous-section \ref{subsec:jacobi}. 

\subsection{Quelques éléments pour le calcul de \(K\)}
\label{subsec:calculK}

\indent Le calcul des éléments de K est fait de la manière usuelle, par assemblage des contributions des éléments pour les coefficients \(k_{ij}\). On précise dans la liste suivante quelques détails de l'implémentation de ce calcul : 

\begin{itemize}
	\item Le gradient de \(\phii\) sur l'élément \(T\) est donnée par \((\nabref \phii)^T = \frac{\normT{i}}{d\text{!}|T|}\), où \(|T|\) est l'aire de \(T\), \(d=2\) est le nombre de dimensions spatiales et \(\normT{i}\) est le vecteur entrant à \(T\), dans le côté opposé à \(i\) et de norme égale à la longueur de ce côté \cite{vecNormal}.
	\item La fonction \(\omega\) est considérée constante dans chaque élément \(T\) et égale à la moyenne \(\omega^T\) de sa valeur sur les noeuds de \(T\).
\end{itemize}

\indent On a, ainsi : 

\begin{equation}
\label{eq:calculK_2d}
\begin{gathered}
\begin{aligned}
	k_{ij} & = \iDomh{ \omega \nabref \phii  \cdot \nabref \phij } = \sum_{T \ni i} {\iT{ \omega \nabref \phii \cdot \nabref \phij }} = \\
	       &  = \sum_{T \ni i}
	              { 
	                     { |T|\omega^T \frac{\normT{i}\cdot \normT{j}}{4|T|^2}
	                     }
	              }
	          = \sum_{T \ni i}
	              { 
	                     { \omega^T \frac{\normT{i}\cdot \normT{j}}{4|T|}
	                     }
	              }	              
\end{aligned}
\end{gathered}
\end{equation}

\indent \textbf{Remarque 1} : comme montre le développement du modèle fait ci-dessus, le calcul de l'intégrale qui donne les éléments de la matrice \(K\) est \textbf{toujours fait sur le maillage de référence}, ce qui implique que, dans \eqref{eq:calculK_2d}, on utilise toujours les vecteurs normaux et les aires du maillage initial. Néanmoins, \textbf{la fonction \(\omega\) est actualisée à chaque itération dans le maillage physique}, afin de bien exprimer l'évolution du maillage au cours des itérations, indiquant si, dans leurs nouvelles positions, les nœuds sont dans des régions de raffinement ou pas.

\indent

\indent \textbf{Remarque 2} : Pour le calcul des éléments de \(K\), on a profité du fait que, dans la méthode d'éléments finis P1, on a \(k_{ij} = 0\) si les noeuds \(i\) et \(j\) n'appartient pas au même élément. Ainsi, la matrice est creuse, ce qui nous a motivé à le stocker avec une structure adaptée (au lieu d'allouer une matrice de taille \(N \times N\)), afin de réduire la consommation d'espace mémoire, comme décrit ci-dessous : 


\begin{itemize}
	\item On identifie, d'abord, le nombre maximal de voisins d'un noeud du maillage, \(\Nn\), sous la convention qu'un noeud est toujours voisin de lui-même;
	\item On alloue deux vecteurs :  
	\begin{itemize}
		\item Un vecteur d'entiers, \(A\), de taille \((\Nn+1).(N+1)\);
		\item Un vecteur de doubles, \(\Ks\) , de taille \(\Nn.(N+1)\);
	\end{itemize}
%	\item Les éléments de ces vecteurs qui se réfèrent à l'élément \(i\) sont : 
%	\begin{itemize}
%		\item Dans \(A\) : \(A_{(\Nn+1).i}\) jusqu'à \(A_{(\Nn+1).(i+1)-1}\);
%		\item Dans \(\Ks\) : \(\Ks_{\Nn . i}\) jusqu'à \(\Ks_{\Nn  .(i+1)-1}\);
%	\end{itemize}
	\item Le vecteur \(A\) contient les index des voisins de chaque noeud. Par convention,
	\begin{itemize}
		\item \(A_{(\Nn+1).i} = \Nn^i\) contient le nombre de voisins de \(i\);
		\item \(A_{(\Nn+1).i+1}\) jusqu'à \(A_{(\Nn+1).i+\Nn^i}\) contiennent les index des voisins de \(i\) (pour commodité, on garde toujours \(A_{(\Nn+1).i+\Nn^i} = i\), mais cela n'est pas nécessaire)
	\end{itemize}
	\item Le vecteur \(\Ks\) contient les éléments de la matrice : 
	\begin{itemize}
		\item Si \(A_{(\Nn+1).i + z} = j\), alors \(\Ks_{\Nn . i + z - 1} = K_{ij}\)
	\end{itemize}
\end{itemize}

\indent Le stockage ici proposé n'est pas optimal, car, dans des maillages non structurés, les noeuds n'ont pas tous le même nombre de voisins. Par ailleurs, les \(\Nn+1\) premier éléments de \(A\) et les \(\Nn\) premiers éléments de \(\Ks\) ne sont pas utilisés, pour être en concordance avec les conventions d'index utilisés dans la bibliothèque MMG. Néanmoins, par rapport au stockage de la matrice complète, on a vérifié des très grandes avantages : 

\begin{itemize}
	\item Par exemple, dans un test avec environ \(28000\) noeuds et 165 mille élements non nulles dans la matrice \(K\), on est passé d'un stockage de 783 million de doubles (soit 0.02\% d'utilisation) à un stockage de 308 mille doubles (soit 54\% d'utilisation) et 336 mille entiers, avec un considérable gain en temps d'exécution.
	\item On a observé aussi des gains en temps d'exécution du programme. Dans la formulation originale, sans garder les index des voisins de chaque noeud, il fallait appeler une fonction qui retourne la liste de voisins. Cet appel était fait une fois par itération pour chaque noeud, lors de la résolution du système linéaire.
\end{itemize}


\subsection{Résolution du système linéaire}
\label{subsec:jacobi}

\indent Le système linéaire \eqref{eq:syst_final} est résolu de façon itérative, avec la méthode de Jacobi. Dans les tests, on utilise en général un nombre petit d'itérations (dix ou vingt), ce qui donne de bons résultats pour un temps de calcul raisonnable. La solution est calculée en terme des déplacements \(\vecdelta = \vecx - \vecxi\). Ainsi, on réécrit le système linéaire \eqref{eq:syst_final} sous la forme 

\begin{equation*}
	Kx = K(x-\delta+\delta) = 0 \longrightarrow K\delta = -K\xi
\end{equation*}

\indent Ainsi, \(\forall i \in \{1,...,N\} \) :

\begin{equation*}
	k_{ii}\delta_i^{[n+1]} = -\sum_{j=1,j \neq i}^N k_{ij}\delta_{j}^{[n]} - \sum_{j=1}^N k_{ij}\xi_{j} = -\sum_{j=1}^N k_{ij}(\xi_j + \delta_{j}^{[n]}) + k_{ii}\delta_i^{[n]}
\end{equation*}

\indent Donc, finalement,

\begin{equation*}
	\delta_i^{[n+1]} =  \delta_i^{[n]} -\frac{1}{k_{ii}} \sum_{j=1}^N k_{ij}(x_j^{[n]})
\end{equation*}

\indent \textbf{Remarque} : avant d'actualiser la position des points, il faut vérifier si le déplacement calculé ne conduit pas à un croisement des noeuds. Pour faire cette vérification, on calcule les aires signées des éléments (i.e., en considérant l'ordre des noeuds). Si nécessaire (\emph{i.e.}, si un des éléments a une aire négative), on relaxe le déplacement global: 

%\begin{equation}
%	x_i^{[n+1]} = \xi_i + \theta \delta_i^{[n+1]}
%\end{equation} 

\begin{equation*}
	\vecx^{[n+1]} = \vecx^{[n]} + \theta \left( \vecxi + \vecdelta^{[n+1]} - \vecx^{[n]} \right)
\end{equation*} 

\noindent avec \(\theta \in [0,1]\). La relaxation est ainsi appliquée sur la différence entre deux positions successives, non pas sur le déplacement par rapport à la position initiale, car dans ce cas-ci on peut avoir des ``retours en arrière" des points du maillage.
