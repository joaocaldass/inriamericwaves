\section{Introduction}

\indent L'objectif de l'adaptation de maillages développée dans ce stage est de modifier la position de ses noeuds par rapport à une fonction donnée, ou à une solution d'un problème physique, de façon qu'on puisse avoir un raffinement variable au long du domaine et qui représente bien la présence de fortes gradients, surfaces, etc., sans modifier le nombre de points ni la connectivité du maillage.

\indent Dans ce rapport, on présent dans un premier moment l'application de l'algorithme à l'adaptation à des fonctions Level Set, à fin d'obtenir une bonne représentation d'un objet. Cette fonction, définie pour tout point du maillage, est la distance signée entre le point et l'objet, où la signe indique s'il est à son intérieur ou extérieur \cite{ducrot}. Ainsi, la ligne de niveau 0 de la fonction Level Set représente la surface de l'objet, et on utilise ce fait pour orienter l'adaptation du maillage.

\indent Ensuite, on résout des problèmes de la mécanique des fluides sur les maillages adaptés et, avec les résultats obtenus, on fait des nouvelles adaptations, mais cette fois-ci en utilisant au même temps la fonction Level Set et la solution physique du problème. Avec cette procédure, on est capable d'obtenir des maillages encore plus appropriées au calcul envisagé. 

\indent Le rapport est organisé de la façon suivante : dans la section \ref{sec:modele}, on présente d'abord, de façon plus générale, le modèle utilisé pour l'adaptation de maillages, selon la formulation développée par \cite{arpaia}, et des détails concernant son implémentation. L'application du modèle à l'adaptation à des fonctions Level Set est décrite dans la section \ref{sec:application}, avec quelques exemples de tests réalisés et une indication des paramètres qui ont produit les meilleurs résultats. Le couplage avec l'adaptation physique et les résultats obtenus sont présentés dans la section \ref{sec:adapPhysique}. Enfin, dans la section \ref{sec:nonstat}, on utilise le modèle pour adapter le maillage à des objets en mouvement. Par ailleurs, on remarque que la plupart du contenu de ce rapport se réfère à des cas 2D; pourtant, nous présentons aussi quelques résultats pour des exemples 3D.

\indent Les exemples présentés dans ce rapport ne sont qu'une petite partie de l'ensemble des tests réalisées au cours de ce stage. Ces tests ont eu une grande importance pour valider, corriger et développer le modèle, tester la bibliothèque et les plusieurs parties du code, et trouver les paramètres et stratégies d'adaptation qui nous permettent d'obtenir les meilleurs résultats en tenant compte des objectifs décrits ci-dessus.