\section{Présentation de l'organisme d'accueil}
\label{sec:organisme}

\subsection{Inria}
\label{subsec:inria}

\indent L'Institut National de Recherche en Informatique et en Automatique (INRIA) est un établissement public de recherche dans les sciences du numérique. Créé en 1967, il compte aujourd'hui huit centres dans le territoire français. 

\begingroup
\centering
\includegraphics[scale=.3]{figures/logos/Inria.jpg}
\captionof{figure}{Logo de l'Inria}
\endgroup

\indent Depuis la création de l'Inria, ses activités sont organisées autour du modèle d'équipe-projet, avec environ vingt personnes chacune, comprenant un "leader scientifique", des chercheurs, enseignants chercheurs, doctorants, ingénieurs et stagiaires. Chaque équipe a des objectifs de recherche bien définis, et son travail se déroule au long de quatre ans, avec des possibles prolongations en dépendant de l'évaluation faite par l'Inria et des experts internationaux extérieurs. Actuellement, il y a 178 équipes-projet dans l'institut, divisées dans cinq domaines : 

\begin{itemize}
	\item Mathématiques appliquées, calcul et simulation;
	\item Algorithmique, programmation, logiciels et architectures;
	\item Réseaux, systèmes et services, calcul distribué;
	\item Perception, cognition, interaction;
	\item Santé, biologie et planètes numériques
\end{itemize}

\indent On peut ainsi voir que la recherche développée à l'Inria couvre les plus variés domaines des mathématiques appliqués et de l'informatique, y compris plusieurs thèmes multidisciplinaires. On justifie ainsi son slogan : "Inventeurs du monde numérique". Ces activités se déroulent sous plusieurs partenariats avec l'industrie, les entreprises et le monde académique, sous la forme de brevets et logiciels, création de start-ups et accueil de doctorants et enseignants, ce qui attribue à l'institut une grande importance en contexte français et international.

\indent Parmi les partenariats dans l'industrie, Inria a notamment des relations avec Google, Microsoft, EDF, Astom et Total. Dans le cadre académique, il y a plus de trente partenariats avec les plus importantes universités, grandes écoles et écoles d'ingénieurs de la France, et aussi avec d'autres organismes publics de recherche, comme le Centre National de la Recherche Scientifique (CNRS) et le Commissariat à l'Énergie Atomique (CEA). De plus, la présence d'Inria à l'étranger est à chaque fois plus forte, ayant des collaborations, laboratoires et centres de recherche dans des plusieurs pays, comme le Chili et le États-Unis.

%\indent En travaillant à Inria, j'ai pu voir et vivre toute une ambiance qu'inspire et motive la recherche scientifique. Les chercheurs et doctorants de l'équipe étaient toujours disponibles pour m'orienter, répondre mes questions et m'aider à conduire mon travail. De plus, étant hétérogène le travail au sein de chaque équipe, avec plusieurs thèmes, thèses et projets informatiques se développant et convergeant vers l'objectif scientific commun, j'ai eu une certaine liberté pour choisir mes principaux activités, selon mes connaissances, mes préférences et les thèmes que j'avais envie d'apprendre et améliorer.

\subsubsection{Inria Bordeaux Sud-Ouest}

\indent Plus précisément, mon premier stage a eu lieu au Centre de Recherche Inria Bordeaux Sud-Ouest, situé à Talence, ville voisine à Bordeaux. Accueillant 21 équipes-projet distribués parmi les cinq domaines de recherche définis par Inria, ce centre a un fort partenariat avec l'Université de Bordeaux, notamment l'Institut de Mathématiques de Bordeaux (IMB), situé également à Talence.

\subsection{L'équipe de travail}

\indent Ce premier stage s'est déroulé sous l'orientation de Cécile DOBRZYNSKI, maître de conférences de l'Institut de Mathématiques de Bordeaux (Université de Bordeaux I) et chercheuse à l'équipe CARDAMOM, et de Mario Ricchiuto, leader de l'équipe. Par ailleurs, j'ai aussi travaillé avec Léo NOUVEAU, qui fait sa thèse à l'équipe, sur la résolution des équations de Navier Stokes avec des méthodes de pénalisation et l'utilisation de l'adaptation de maillage.

%\subsection{La maître de stage : \CD} 
%\label{subsec:cecile}
%
%\indent Ayant un formation en Mathématiques Appliquées et Ingénierie Mathématique, avec emphase en mécanique, à l'Université Paris-Sud Orsay, à l'Université Pierre \& Marie Curie et à l'Université de Louvain (Belgique), ma maître de stage, \CD, est actuellement maître de conférences à l'École Nationale Supérieure d'Électronique, Informatique, Télécommunications, Mathématique et Mécanique de Bordeaux (ENSEIRB-MATMÉCA) et à l'Institut de Mathématiques de Bordeaux (IMB - Université de Bordeaux I). \citep{cecileCV}
%
%\indent Ses principaux thèmes de recherche sont le déplacement de corps rigides, l'adaptation de maillages et l'aérothermique des bâtiments. Elle a rejoint l'INRIA lors de sa thèse de doctorat ("Adaptation de maillage anisotrope 3d et application à l'aérothermique des bâtiments"), sous la direction d'Olivier PIRONNEAU et Pascal FREY. Le travail avec ce dernier a donné origine à des logiciels open-source d'optimisation de maillage (MMG2D, MMG3D, MMGS) \citep{mmg}. 