\begingroup
\centering

\includegraphics[scale=.075]{figures/logos/enpc.png}


\noindent {\large École des Ponts ParisTech}

\vspace*{1.\baselineskip}

\noindent {\normalsize 2015-2016}

\vspace*{1.\baselineskip}

\noindent {\Large Rapport de stage long (fractionné)}

\vspace*{3.\baselineskip}


\noindent {\large Joao Guilherme CALDAS STEINSTRAESSER}

\vspace*{.2\baselineskip}

\noindent {\large Élève en double diplôme - Ingénierie Mathématique et Informatique (IMI)}

\vspace*{3.\baselineskip}

\noindent {\Large Modèle d'adaptation de maillage à des surfaces et variables physiques appliqué à des problèmes de la mécanique des fluides}

\vspace*{.2\baselineskip}

\noindent {\normalsize Stage réalisé au sein de l'équipe CARDAMOM - Inria Bordeaux Sud-Ouest}

\vspace*{.2\baselineskip}

\noindent {\normalsize  200 Avenue de la Vieille Tour, 33405 Talence, France}

\vspace*{.2\baselineskip}

\noindent {\normalsize  Juillet - Décembre 2015}

\vspace*{.2\baselineskip}

\noindent {\normalsize  Maître de stage : Mme Cécile DOBRZYNSKI}

\vspace*{1.5\baselineskip}

\noindent {\Large Méthode de décomposition de domaine appliquée à la résolution de modèles non linéaires et/ou dispersifs pour la propagation des ondes}

\vspace*{.2\baselineskip}

\noindent {\normalsize Stage réalisé à MERIC / Inria Chile}

\vspace*{.2\baselineskip}

\noindent {\normalsize Avenida Apoquindo 2827, piso 12 - Las Condes – Santiago, Chile}

\vspace*{.2\baselineskip}

\noindent {\normalsize  Mars - Août 2016}

\vspace*{.2\baselineskip}

\noindent {\normalsize  Maître de stage : M. Antoine ROUSSEAU}

\vspace*{\fill}

\endgroup