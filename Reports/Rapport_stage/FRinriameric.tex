\subsection{Présentation des organismes d'accueil}

\indent Le bon déroulement du premier stage de mon année de césure m'a guidé dans la recherche du deuxième stage. En cherchant surtout à continuer à travailler dans les domaines de recherche en mathématiques appliquées, et aussi sous la bonne ambiance de travail à Inria, je suis allé au Chili pour faire le stage dans une association entre les institutions Inria Chile et MERIC (Marine Energy Research \& Innovation Center).

\subsubsection{Inria Chile}

\indent L'Inria Chile est un des résultats des efforts de l'Inria dans les dernières années pour installer des laboratoires internationaux et de former et renforcer des partenariats et collaborations scientifiques avec des universités et entreprises à l'étranger. Considérée une région stratégique par Inria, l'Amérique compte avec trois \emph{"Laboratoires internationaux"}, deux aux États-Unis et un au Chili.

\begingroup
\centering
\includegraphics[scale=.3]{figures/logos/Inria-Chile.png}
\captionof{figure}{Logo de l'Inria Chile}
\endgroup

\indent Inria Chile a été crée en 2012 avec le support de CORFO (\emph{Corporación de Fomento de Producción}), une agence de l'état chilien responsable pour la promotion de l'investissement, l'innovation et l'entreprenariat. Étant un centre de transfert technologique dans les technologies de l'information et de  l'information au Chili, cet support se déroule sous le "programme d'attraction de centres d'excellence internationale pour la compétitivité" de CORFO.

\subsubsection{MERIC}

\indent Inria Chile partage ses bureaux avec MERIC, un centre de excellence qui a pour objectif la recherche, le développement technologique et l'innovation liés à l'énergie marine, afin de créer une référence chilienne et internationale dans ce sujet. Crée par l'entreprise française de défense navale DCNS, MERIC est financé par CORFO et le Ministère d'Énergie chilien et déroule ses activités en collaboration avec Inria Chile, des universités chiliennes (Pontificia Universidade de Chile et Universidade Austral de Chile), des entreprises d'énergie (Enel Green Power Chile et Chilectra) et des instituts de promotion de l'innovation (Fundación Chile).  

\begingroup
\centering
\includegraphics[scale=.3]{figures/logos/meric.png}
\captionof{figure}{Logo de MERIC}
\endgroup

\indent L'énergie marine a une très grande importance dans le contexte chilien et aussi dans le contexte mondial. Pour le Chili, avec son côte de plus de 6000 km de longueur, le mouvement des marées et des ondes constituent un énorme potentiel énergétique, étant ainsi une question stratégique pour le pays. Par ailleurs, le Chili est inséré dans un contexte spécial concernant les phénomènes naturels extrêmes : l'intense activité sismique de la région fait les tremblements de terre et les tsunamis être une réalité dans l'histoire et le quotidien du pays, en demandant des études spécifiques pour que la production d'énergie marine puisse être mise en pratique sans risque et durable.

\indent Dans un point de vue plus global, l'énergie marine, étant une forme d'énergie renouvelable, est une des protagonistes dans la construction du développement durable. Néanmoins, elle est encore une très jeune thématique d'étude, en exigeant plusieurs efforts et recherches pour le développement et réduction des coûts des technologies associées, pour permettre son effective implémentation et remplacement des formes d'énergie traditionnelles et limitées. En plus, il est essentiel de tenir en compte l'insertion des formes d'énergie dans le contexte du développement durable ne comprenne seulement son aspect renouvelable : il faut toujours étudier les conséquences environnementales et sociales de son production, en cherchant à minimiser les impacts sur les habitats, ses formes de vie et les populations humaines.

\indent Ainsi, le travail développé à MERIC a un caractère très multidisciplinaire, en englobant les plus variés sujets liés à la production d'énergie marine, groupés dans deux lignes de recherche et développement : "Développement de lieux y connaissance de l'environnement chilien" et "Développement de technologies au Chili".

\indent La première ligne de recherche envisage à étudier les potentiaux lieux de production d'énergie marine au Chili. Dans cet étude, il est compris des aspects physiques physiques, chimiques, biologiques et sociaux, sous des projets concernant, par exemple, la caractérisation des sites en potentiel pour l'implémentation d'énergie marine, la corrosion marine, la déposition de matériel biologique (\emph{biofouling}), les mammifères marines, l'évaluation des impacts sociaux et environnementaux et la modélisation mathématique pour la production d'énergie marine.

\indent La deuxième ligne de recherche, à son tour, étudie l'installation, manutention et utilisation de technologies pour la production d'énergie marine, avec des projets qui comprennent le développement de technologies, son adaptation 'a l'environnement chilien (notamment dans ce qui concerne les risques naturels), le développement d'un centre de tests pour la validation de la recherche en énergie marine, l'étude de l'état d'art de la législation chilienne concernant la production d'énergie marine, et le design d'usines de dessalement.

\indent Mon travail réalisé dans ce stage s'insérait dans la première ligne de recherche, dans le projet "Modélisation avancée pour l'énergie marine". Ainsi, ma mission ne correspondait qu'à une petite fraction de tous les aspects liés à la production d'énergie marine et étudiés par MERIC.
