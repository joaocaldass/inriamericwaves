\section{Conclusion}

\indent Le travail développé au long du deuxième stage de mon année de césure a eu deux thématiques principales : l'étude et implémentation numérique de modèles non linéaires de propagation des ondes; et le développement d'une méthode de décomposition de domaine permettant de résoudre un de ces modèles.

\indent Dans la première ligne, on a travaillé sur l'équation de KdV et les équations de Serre. Les études de ces deux modèles ont eu plusieurs points en commun : notamment, dans les deux cas on a proposé la résolution par des méthodes de \emph{splitting}, en séparant les termes d'advection et ceux de dispersion, et on a implémenté pour chaque pas du \emph{splitting} les mêmes types de schéma numérique (volumes finis pour le pas d'advection et différences finis pour le pas de dispersion, dans le cas de problèmes non périodiques). Néanmoins, les différentes complexités de ces modèles ont posé des difficultés distinctes pour l'étude de chacun d'eux.

\indent On a d'abord implémenté l'équation de KdV, qui a une forme relativement simple : étant constitué d'une seule équation, ce modèle a comme solution un scalaire et, par ailleurs, les équations résultants du \emph{splitting} peuvent être discrétisées de façon aussi simple. Ainsi, le travail sur ce modèle a avancé très rapidement, et sa simplicité nous a permis aussi de l'étudier de façon plus détaillée, avec par exemple une analyse d'échelle. On n'a pas trouvé d'expression de son solution analytique, mais le modèle a été bien validé avec des exemples trouvés dans la littérature. Ainsi, l'équation de KdV a été une enrichissante introduction à l'étude des modèles non linéaires et dispersives. 

\indent Ensuite, on s'est lancé au modèle de Serre, qui est beaucoup plus complexe. Il s'agit d'un système de deux équations, avec des solutions constituées par la hauteur de l'eau et la vitesse. Les pas du \emph{splitting} demandent des discrétisations moins évidentes que celles pour l'équation de KdV, même dans le cas du pas de dispersion, qui ramène à une seule équation (car la hauteur d'eau est constante en temps). Par ailleurs, on a dû travailler notamment sur l'ordre des schémas. Plusieurs semaines d'étude, tests, débougage et discussions avec des chercheurs en Chile et en France se sont déroules avant qui nous sommes arrivés à la conclusion que des schémas d'ordre 4 sont nécessaires (pour l'advection et la dispersion), ce qui nous a finalement permis de valider notre implémentation numérique avec la solution analytique.

\indent Parallèlement, on a étudié et implémenté une méthode de décomposition de domaine, ce qui est plus directement lié à la ligne de recherche de MERIC et dont le travail développé à eu par résultat la production d'un papier scientifique, qu'on a envoyé par publication, qu'on espère que soit accepté.

\indent Ce travail a été développé sur l'équation de KdV (plus précisément sur sa forme linéarisée, sans le terme d'advection), qui a une forme plus simple et pour laquelle il y a dans la littérature des études concernant les conditions aux limites transparentes. Ces conditions ont joué un rôle très importante dans notre travail. On les a proposé des approximations et on les a utilisées comme des conditions à l'interface (IBCs) entre les subdomaines, lors de l'implémentation de la DDM.

\indent On n'avait pas l'objectif de proposer des TBCs meilleures que celles formulées dans les papiers dont on s'est basé. En fait, nos TBCs approximées sont moins précises, mais on a vérifié qu'elles fournissent des résultats raisonnables et, notamment, qui elles ont une forme et une implémentation très simples (étant écrite en fonction de seulement un coefficient) et fournissent des convergences rapides de la méthode de Schwarz.

\indent Par ailleurs, on a proposé aussi des petites corrections à ces IBCs approximées, afin d'assurer que la solution fournie par la DDM converge exactement vers notre solution de référence (la solution du même problème calculée dans le monodomaine). Finalement, on a vérifié que la vitesse de convergence dépende du pas de temps, de la taille de la maille et du coefficient pour la construction des IBCs; ainsi, à partir d'un processus d'optimisation, on a trouvé et validé des expressions de régression qui permettent de calculer le coefficient optimal (\emph{i.e.}, celui qui fournit la convergence la plus rapide) en fonction de $\Delta t $ et $\Delta x$.

\indent À la fin du stage, on a démarré l'étude des DDMs pour les équations de Serre. Plus précisément, en envisageant des applications pratiques, on s'intéresse plutôt au couplage entre ce modèle et les équations de \emph{shallow water} non linéaires (NSWE), ce qui sera un des futurs thèmes de travail de l'équipe. Une autre suite naturelle du travail réalisé dans ce stage serait l'étude des DDMs pour des problèmes en deux dimensions.

\indent En rappelant les objectifs plus généraux de MERIC, les sujets étudiés dans ce stage ont, évidemment, un caractère très introductoire dans ce qui concerne les applications à l'étude de l'énergie marine. Néanmoins, en partant de modèles plus simples et en développant et validant des méthodes de décomposition de domaine pour eux, on permet d'orienter les prochaines pas dans la ligne de recherche de MERIC liée à la modélisation mathématique, qui peut ainsi considérer des modèles à chaque fois plus complexes et à des situations plus proches des applications réelles.