\section{BBM equation}
\label{sec:BBM}

\subsection{The model}

\indent The second model for wave propagation that we consider in this project is the BBM equation, which models a long-wave propagation accounting for nonlinear and dispersive effects (in an alternative formulation for the KdV equation, with better stability and numerical properties). This equation is derived by \cite{BBM1971} and reads

\begin{equation}
    u_t + u_x + (u^2)_x - u_{xxt} = 0 \ , \ \ x \in [x_{min},x_{max}], \ \ t \in [0, t_{max}] \\
\end{equation}

\subsection{Discretization}

\indent The BBM model will be treated here in the same way we did for the KdV equation : we will use an split method, leading to the resolution of the following problem in each time step: 

\begin{equation}
\begin{cases}
    v_t + v_x + (v^2)_x = 0 \ \ ,\ t \in [t_n,t_{n+1}], \  v(t_n) = u(t_n) \\
    w_t - cw_{xxt} = 0 \ \ , \ t \in [t_n,t_{n+1}], \  w(t_n) = v(t_{n+1}) \\
    u(t_{n+1}) = v(t_{n+1})
\end{cases}
\end{equation}

\noindent given an initial solution and appropriate boundary conditions.

\indent The first equation (which is exactly the same as in the KdV equation) will be solved with a Finite Volume method, with a 4th order Runge-Kutta discretization for the time derivative, as described in the section \ref{sec:KdVSplitted1}.  The second equation will be solved with a Fourier spectral method in the periodic case, and a finite difference method in the nonperiodic one, as described below :

\paragraph{Periodic case}

\indent The second equation of the BBM splitted model can be written as

$$(w - cw_{xx})_t=0$$

\noindent showing that $w - w_{xx}$ does not depend on time. Therefore, for each time step $[t_n,t_{n+1}] : $

$$ \label{eq:dtzero}w - cw_{xx} = (w - cw_{xx})\rvert_{t=t_n} = (v - cv_{xx})\rvert_{t=t_{n+1}} = g(x)$$

\indent This equation will be solved using the Fourier method. Let $\hat{w}(k,t_n)$ and $\hat{g}(k)$ be the Fourier coefficients of $w(x,t_n)$ and $g(x)$ respectively.  The Fourier transform of the above equation gives

$$(\hat{w})(k,t) = \frac{\hat{g}(k)}{1+ck^2}$$

\indent The right-hand side of the last equation does not depend on time. Therefore, the inverse Fourier transform using the coefficients $\hat{w}(k,t)$ gives $w(x,t_{n+1})$

\indent Though the simplicity of this implementation, we did not obtained stable numerical solutions. As we moved to the study of the Serre equation, following the objectives in this project, the numerical resolution of the BBM equation remained as an open question.

\paragraph{Nonperiodic case}

\indent We propose for the nonperiodic case a Finite Difference discretization of the equation \eqref{eq:dtzero}, which is a second-order ODE, leading to the resolution of the linear system

$$Aw^{n+1} = g$$

\indent Nevertheless, there is an evident problem with this approach. The matrix $A$ is constructed with a Finite Difference approximation for the operator $1 + \partial_{xx}$, and, to construct the right hand side of the linear system, we must approximate  $ g = (v - cv_{xx})\rvert_{t=t_{n+1}} $, which is the same operator applied to $v\rvert_{t=t_{n+1}}$. In other words, we will have a system of the form

$$ \label{eq:systemBBM} Aw^{n+1} = Av^{n+1}$$,

\noindent and, as A is nonsingular, we conclude that $w^{n+1} = v^{n}$, which means that the dispersive part of the BBM equation does not modify the solution. Evidently, this is not necessarily true if we modify the matrix in the left side of \ref{eq:systemBBM} to take in account the boundary conditions, but if we consider an example in which the domain is much larger than the support of the initial solution (as we have done for the nonperiodic test cases for the KdV equation), this problem is observed.
