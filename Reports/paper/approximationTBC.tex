\section{Approximations for the Transparent Boundary Conditions}

\indent The computation of the TBCs \eqref{eq:continuousTBC1} - \eqref{eq:continuousTBC3} is not simple due to the inverse Laplace transform that must be performed, which makes these conditions to be nonlocal in time. Therefore, we will propose approximations of the root \eqref{eq:lambda} that avoid integrations in time, making the TBCs very simple.

\indent Obviously, as we can see through the results shown in this section, the resulting boundary conditions are not so precise as the ones proposed by \cite{besse2015} (who propose TBCs derived for the discrete linearized KdV equation). Nevertheless, the objectives of our work and the work of \cite{besse2015} are very different : while they seek to minimize the error of the computed solution due to the boundary conditions, we want to propose here a Domain Decomposition Method ................................

\indent We firstly notice that we can rewrite \eqref{eq:continuousTBC1}-\eqref{eq:continuousTBC3} as

\begin{equation}
\label{eq:TBC2}
    \begin{cases}
        u(t,a) - U_2 \laplinv \left( \frac{\lambda_1(s)^2}{s} \hat{u}_x(t,s) \right)  - U_2 \laplinv \left( \frac{\lambda_1(s)}{s}  \hat{u}_{xx}(t,s) \right) = 0 \\
        u(t,b) - \laplinv \left( \frac{1}{\lambda_1(s)^2}   \hat{u}_{xx}(t,s) \right) = 0 \\
        u_x(t,b) - \laplinv \left( \frac{1}{\lambda_1(s)}   \hat{u}_{xx}(t,s) \right) = 0 
    \end{cases}
\end{equation}

\noindent where$\hat{u}(s,x) = \mathcal{L}u(t,x)$ the Laplace transform in $t$ of $u$.  Moreover, the inverse Laplace transform satisfies the following properties :

\begin{itemize}
	\item Linearity :
		\begin{equation}
			\label{eq:linearityLaplace}
				\laplinv \left[a_1\hat{u}_1(s,x) + a_2\hat{u}_2(s,x)\right] = a_1u_1(t,x) + a_2u_2(x,t)
		\end{equation}
	\item Derivative :
		\begin{equation}
			\label{eq:derivativeLaplace}
			\laplinv \left[ s\hat{u}(s,x) \right] = u_t(s,t) + \laplinv \left[ u(0,x) \right] =  u_t(s,t) +  u(0,x) \delta (t)
		\end{equation}
	\item Convolution :
	\begin{equation}
		\label{eq:convolutionLaplace}
		\laplinv \left[ \hat{u}_1(s,x)\hat{u}_2(s,x)\right] = \laplinv \left[ \hat{u}_1(s,x)\right] * \laplinv \left[ \hat{u}_2(s,x)\right]
	\end{equation}
\end{itemize} 

\noindent where $\delta (t)$ is the Dirac delta function and $*$ denotes the convolution operator.

\indent The properties \eqref{eq:linearityLaplace} to \eqref{eq:convolutionLaplace} motivate us to approximate the operands of the inverse Laplace transforms in \eqref{eq:continuousTBC1}-\eqref{eq:continuousTBC2} by polynomials in $s$. Therefore, we use initially a constant polynomial for this approximation. Before that, we remark the following useful relations between the mentioned operands, as a consequence of \eqref{eq:lambda}

\begin{gather}
	\label{eq:operands1}
	\frac{\lambda}{s}  = -\left( \frac{\lambda^2}{s} \right)^2 \\
	\label{eq:operands2}
	\frac{1}{\lambda_1(s)^2}  = \left( \frac{\lambda^2}{s} \right) \\
	\label{eq:operands3}
	 \frac{1}{\lambda_1(s)}  = -\frac{\lambda^2}{s}
\end{gather}

\subsubsection{Approximation of the TBCs using a constant polynomial}

\indent Using \eqref{eq:operands1}-\eqref{eq:operands3}, we will use the constant polynomial $P_0(s) = c$ for the following approximations :

\begin{gather}
	\label{eq:appP0A}
	\frac{\lambda^2}{s}  = c \\
	\label{eq:appP0B}
	\frac{\lambda}{s}  = -c^2 \\
	\label{eq:appP0C}
	\frac{1}{\lambda_1(s)^2}  = c^2\\ 
	\label{eq:appP0D}
	 \frac{1}{\lambda_1(s)}  = -c 
\end{gather}

\indent Replacing in \eqref{eq:continuousTBC1}-\eqref{eq:continuousTBC2} and considering possibly different polynomial approximations for the left and the right boundaries (respectively with the coefficients $c_L$ and $c_R$), we get the approximate Transparent Boundary Conditions : 

\begin{equation}
\label{eq:appTBCP0}
    \begin{cases}
        u(t,a) - c u_x(t,a)  + c^2  u_{xx}(t,s) = 0 \\
        u(t,b) - c^2    \hat{u}_{xx}(t,s) = 0 \\
        u_x(t,b) + c u_{xx}(t,s)= 0 
    \end{cases}
\end{equation}

\indent Using finite difference approximations, (\ref{eq:appTBCP0}) is discretized as

\begin{equation}
\label{eq:appDiscTBCP0}
    \begin{cases}
        u_0 - c_L \frac{u_1 - u_0}{\Delta x}  + c_L^2  \frac{u_0 -2u_1 + u_2}{\Delta x^2} = 0 \\
        u_N - c_R^2    \frac{u_N -2u_{N-1} + u_{N-2}}{\Delta x^2} = 0 \\
        \frac{u_N - u_{N-1}}{\Delta x}  + c_R^2    \frac{u_N -2u_{N-1} + u_{N-2}}{\Delta x^2} = 0 
    \end{cases}
\end{equation}

\subsubsection{Initial numerical experiments}

\indent In order to validate our approximation, observe the general its general behavior when varying the constant approximations $c_L$ and $c_R$ and compare our results with the ones obtained by \cite{besse2015}, we will solve the same numerical test presented in his paper. This problem was originally treated by \cite{zheng2008}, in the study of numerical solutions for the KdV equation.

\begin{gather}
\label{eq:testCaseBesse1}
 u_t + u_{xxx} = 0, \ \ x \in \mathbb{R} \\
 \label{eq:testCaseBesse2}
 u(0,x) = e^{-x^2}, \ \ x \in \mathbb{R}  \\
 \label{eq:testCaseBesse3}
 u \rightarrow 0, \ \ |x| \rightarrow \infty
\end{gather}

\indent The fundamental solution of (\ref{eq:testCaseBesse1}) is

\begin{equation}
    E(t,x) = \frac{1}{\sqrt[3]{3t}}Ai\left(\frac{x}{\sqrt[3]{3t}} \right)
\end{equation}

\noindent where $Ai$ is the Airy function. The exact solution for the problem (\ref{eq:testCaseBesse1}) - (\ref{eq:testCaseBesse3}) is

\begin{equation}
    u_{exact}(t,x) = E(t,x) * e^{-x^2}
\end{equation}

\indent The problem will be solved in the spatial domain $[-6,-6]$

\indent For a quantitative evaluation of the results, we will calculate the same errors defined in the paper of \cite{besse2015}. For each time step, we compute the relative error

$$e^n = \frac{\left\Vert u_{exact}^n - u_{computed}^n\right\Vert_2}{\left\Vert u_{exact}^n\right\Vert_2}$$

\noindent and, in the whole time interval :

$$ e_{Tm} = \max\limits_{0 < n < T_{max}} (e^n) $$

$$ e_{L2} = \sqrt{ \Delta t \sum_{n=1}^{T_{max}} (e^n)^2 } $$

\indent In order to verify the influence of $c_L$ and $c_R$ on the computed solutions (and possibly identify a range of values that better approximate the TBCs), we made several tests with all the possible pairs $c_L,c_R \in {-10,-1,-0.1,0,0.1,1,10}^2$. The results were classified accordingly to their errors $e_{L2}$ (a criteria based on the error $e_{Tm}$ gives a similar result). The figure \ref{fig:firstTestsP0} shows, for some instants, a comparison between the best, the worst and the exact solution. For naming the worst result, we did not considered the ones in which the numerical solution diverged (following the arbitrary criteria $e_{L2} > 10$). 

\begin{minipage}{.5\linewidth}
\centering
\includegraphics[scale=.5]{figures/firstTestsP0Snap2.png}
\end{minipage}
\begin{minipage}{.5\linewidth}
\centering
	\includegraphics[scale=.5]{figures/firstTestsP0Snap3.png}
\end{minipage}
\begin{minipage}{.5\linewidth}
\centering
\includegraphics[scale=.5]{figures/firstTestsP0Snap4.png}
\end{minipage}
\begin{minipage}{.5\linewidth}
\centering
	\includegraphics[scale=.5]{figures/firstTestsP0Snap5.png}
\end{minipage}
\captionof{figure}{Best and worst solution compared with analytical solution, for the constant polynomial approximation \label{fig:firstTestsP0}}

\indent The table \ref{tab:firstTestsP0} presents the ten smallest $e_{L2}$.

\sisetup{round-mode=places}
\begin{center}
\begin{tabular}{c|c|S[round-precision=4,table-number-alignment =  left]}
	\multicolumn{1}{c|}{$c_L$}  & \multicolumn{1}{c|}{$c_R$} & \multicolumn{1}{r}{$e_{L2}$} \\
	\hline
	1.0 & 1.0 & 0.0946839239675 \\
	1.0 & 10.0 & 0.097288371204 \\
	1.0 & 0.1 & 0.0983932287563 \\
	1.0 & 0.0 & 0.0992063806502 \\
	1.0 & -10.0 & 0.099359192771 \\
	1.0 & -0.1 & 0.100021589665 \\
	1.0 &  -1. & 0.10159265619 \\
	10.0 & 1.0 & 0.347003021321 \\
	10.0 & 0.1 & 0.347379492262 \\
	10.0 & 0.0 & 0.347487945035
\end{tabular}
\captionof{table}{Best results (smallest $e_{L2}$) for the constant polynomial approximation \label{tab:firstTestsP0}}
\end{center}

\indent We notice that the results are much sensitive to the coefficient on the left boundary : for a fixed $c_L$, the error is very similar for every $c_R$. This is a consequence of the fact that the solution of this specific problem is practically constant and equal to zero near $x = 6$, but presents strong variations in time near $x = -6$.

\subsection{Approximation of the TBCs using a linear polynomial}

\indent In a similar way as done above, we approximate  $\frac{\lambda^2}{s}$ by $P_1(s) = ds + c$. Without repeating the details of the derivation, we obtain the following discretized approximate TBCs : 

\begin{equation}
\label{eq:appDiscTBCP0}
	\begin{aligned}
    u_0^{n+1} - \left( \frac{d_L}{\Delta t} + c_L \right) \left( \frac{u_1^{n+1} - u_0^{n+1}}{\Delta x}\right) +   \left( \frac{d_L^2}{\Delta t^2} + \frac{2d_Lc_L}{\Delta t} + c_L^2  \right) \left(  \frac{u_0^{n+1} - 2u_1^{n+1} + u_2^{n+1}}{\Delta x^2} \right)  = \\
        -\frac{d_L}{\Delta t}\left( \frac{u_1^{n} - u_0^{n}}{\Delta x}\right) +  \left( 2\frac{d_L^2}{\Delta t^2} + \frac{2d_Lc_L}{\Delta t}\right) \left(  \frac{u_0^{n} - 2u_1^n + u_2^{n}}{\Delta x^2} \right) +   -  \frac{d_L^2}{\Delta t^2} \left(  \frac{u_0^{n-1} - 2u_1^{n-1} + u_2^{n-1}}{\Delta x^2} \right)
   \end{aligned}
\end{equation} 

\begin{equation}
	\begin{aligned}
    u_N^{n+1} - \left( \frac{d_R^2}{\Delta t^2} + \frac{2d_Rc_R}{\Delta t} + c_R^2  \right) \left(  \frac{u_{N}^{n+1} - 2u_{N-1}^{n+1} + u_{N-2}^{n+1}}{\Delta x^2} \right) = \\
     -\left( 2\frac{d_R^2}{\Delta t^2} + \frac{2d_Rc_R}{\Delta t}\right) \left(  \frac{u_N^{n} - 2u_{N-1}^n + u_{N-2}^{n}}{\Delta x^2} \right) + \frac{d_R^2}{\Delta t^2} \left(  \frac{u_N^{n-1} - 2u_{N-1}^{n-1} + u_{N-2}^{n-1}}{\Delta x^2} \right)
    \end{aligned}
\end{equation} 
   
\begin{equation}
	\begin{aligned}	
    \frac{u_N^{n+1} - u_{N-1}^{n+1}}{\Delta x} + \left( \frac{d_R}{\Delta t} + c_R \right) \left( \frac{u_N^{n+1} -2 u_{N-1}^{n+1} + u_{N-2}^{n+1}}{\Delta x^2}\right) =      \frac{d_R}{\Delta t}\left( \frac{u_{N}^{n} - 2u_{N-1}^{n} + u_{N-2}^n}{\Delta x^2}\right)
    \end{aligned}
\end{equation}

\subsubsection{Initial numerical experiments}

\indent We repeated the numerical tests that we made for the approximation with $P_0$, making the coefficients $c_L$ and $d_L$ assume the values in ${-10,-1,-0.1,0,0.1,1,10}$. In order to avoid a too high computation, and also taking in account the remark we made above about the weak dependence of the results on the coefficients in the right boundary, we realized all the tests assuming $c_R = c_L$ and $d_R = d_L$.

\indent We present in the table \ref{tab:firstTestsP1} the ten best results. As we can see, the best result is the one where $d_L = d_R = 0$ and $c_L = c_R = 1. $, which corresponds to the best results of the approximations using constant polynomials.

	 \sisetup{round-mode=places}
\begin{center}
\begin{tabular}{c|c|S[round-precision=4,table-number-alignment =  left]}
	\multicolumn{1}{c|}{$d_L = d_R$}  & \multicolumn{1}{c|}{$c_L = c_R$} & \multicolumn{1}{r}{$e_{L2}$} \\
	\hline
	0. & 1.0 & 0.0946839239675 \\
	0.1 & 1.0 & 0.12341195 \\
	1.0 & 1.0 & 0.20031603 \\
	10.0 & 0.1 & 0.22037249 \\
	-10.0 & 0.1 & 0.23984486 \\
	10.0 & 1.0 & 0.27161158 \\
	-10.0 &  0.0 & 0.24800816\\
	-10.0 & 1.0 & 0.30039637 \\
	10.0 & 0.0 & 0.27213611 \\
	0.0 & 0.1 & 0.36740764
\end{tabular}
\captionof{table}{Best results (smallest $e_{L2}$) for the linear polynomial approximation \label{tab:firstTestsP1}}
\end{center}

\subsection{Partial conclusion}

\indent 


\subsection{Optimization of the approximate TBCs (minimization of the errors compared to the analytical solution)}

\indent The initial tests made up to here allowed us to validate the approximation of the TBCs proposed by \cite{besse2015} with a constant polynomial in each boundary, guiding us in the following work. In this subsection, we investigate deeply the influence of this approximation over the error of the computed solution, compared with the analytical one. For this purpose, we repeat the computations, but with a much more refined range of coefficients, always using as criteria of quality the error $e_{L2}$ of each test.

\indent As a consequence of the remarks made in the last subsection, this refinement is higher for the coefficient $c_L$. By making a gradual study, in which we identified at each step the intervals of $c_L$ which give the best results and studied it even more deeply (up to a step $0.01$ for the variation of $c_L$), we were able to construct the curves presented in the figure \ref{fig:optimTBC} and identify the empirical optimum $c_L = 1.16$ :

\begin{minipage}{.5\linewidth}
	\includegraphics[scale=.45]{figures/errorOptimOnlyL2.png}
	\captionof{subfigure}{General view of all the tested coefficients}
\end{minipage}
\begin{minipage}{.5\linewidth}
	\includegraphics[scale=.45]{figures/errorOptimOnlyL2Detail.png}
	\captionof{subfigure}{Detail for $c_L \in [0.8,2.0]$}
\end{minipage}
	\captionof{figure}{Error of the numerical solution compared to the analytical solution as function of the constant polynomial approximation for the TBC \label{fig:optimTBC}}